%--------------------------------------------------- %
%--------------------------------------------------- %
\subsection*{1.7 2008 Q3b Logic Networks }

Construct a logic network that accepts as input p and q, which may independently have the value 0 or 1, and
gives as final input $\neg(p \wedge \not q)$ (i.e. $\equiv p \rightarrow q$).\\
\bigskip

\textbf{Logic Gates}
\begin{itemize}
\item AND
\item OR
\item NOT
\end{itemize}
\bigskip

\emph{\textbf{Examiner's Comments:}Many
diagrams were carefully and clearly drawn and well labelled, gaining full
marks. The logic table was also well done by most, but there were a few marks
lost in the final part by failing to deduce that ‘since the columns of the table are
identical the expressions are equivalent’.}

%--------------------------------------------------- %
%--------------------------------------------------- %
\subsection*{1.8 2008 Q3b Logic Networks }
Construct a logic network that accepts as input p and q, which may independently have the value 0 or 1, and
gives as final input $(p \wedge  q) \vee \neg q$ (i.e. $\equiv p \rightarrow q$).



\textbf{Important} Label each of the gates appropriately and label the diagram with a symblic expression for the output after each gate.

