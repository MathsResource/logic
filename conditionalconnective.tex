% http://staff.spd.dcu.ie/breens/102notes/Logic2.pdf
%http://www.proofwiki.org/wiki/Definition:Conditional

The conditional or implication is a binary connective:
\[ p \implies q\]
defined as:
If p is true, then q is true.

This is known as a conditional statement.
A conditional statement is also known as a conditional proposition or just a conditional.

p?q is voiced:
if p then q
or:
p implies q

We are at liberty to write this the other way round. q?p means the same as p?q.
q?p is sometimes known as a reverse implication.

The rest of this article interprets the conditional in its usual mathematical sense.
That is, p?q is considered to be true whenever:
p is false; or
q is true.
%-------------------------------------------------------------%
Answer:
Conditional Connectives
The statement `if p then q' is called a conditional statement and 
is written logically as $p \rightarrow q$.
(This asserts that the truth of p guarantees the truth of q.)
$p \rightarrow q$ can also be read as `p implies q', where p is sometimes 
called the antecedent and q the
consequent.
%-----------------------------------------------%
\textbf{Examples:}
\begin{itemize}
\item[p]: It is raining.
\item[q]: I get wet.
\item[p$\right arrow$q]: If it is raining, then I get wet.
s: It is Sunday.
w: I have to work today.
s ! w: If it is Sunday, then I have to work today.
»s ! w: If it is not Sunday, then I have to work today.
s !»w: If it is Sunday, I do not have to work today.
(s ^ p) !»w: If it is Sunday and it's raining, then I don't have to work today.
To examine the truth or falsity of p ! q, suppose p and q are the following propositions
p: I win the lottery,
q: I will buy you a car.
Then p ! q is the statement `If I win the lottery, then I will buy you a car'.

\newpage


\section{Conditional Connectives}
\begin{itemize}
	\item $p \rightarrow q$ is logically equivalent to $\neg(p \wedge \neg q)$.
\end{itemize}

Construct the truth table for the proposition $p \rightarrow q$.

\begin{center}
\begin{tabular}{|c|c|c|c|}
\hline
p & q & $p \rightarrow q$ & $q \rightarrow p$ \\
\hline
\phantom{sp}0\phantom{sp} & \phantom{sp}0\phantom{sp} & 1& 1 \\
0 & 1 & 1 & 0 \\
1 & 0 & 0 & 1 \\
1 & 1 & 1 & 1 \\
\hline
\end{tabular}
\end{center}

\phantom{sp}
%---------------------------------------------------------%
\section{Conditional Connectives}
Construct the truth table for the proposition $p \rightarrow q$.

\begin{center}
\begin{tabular}{|c|c|c|c|}
\hline
p & q & $p \rightarrow q$ & $q \rightarrow p$ \\
\hline
0 & 0 & 1& 1 \\
0 & 1 & 1 & 0 \\
1 & 0 & 0 & 1 \\
1 & 1 & 1 & 1 \\
\hline
\end{tabular}
\end{center}
\end{document}
%------------------------------------------------------------------ %

