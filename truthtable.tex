
%--------------------------------------------------- %
\documentclass{beamer}
\usepackage{amsmath}
\usepackage{amssymb}
\usepackage{graphicx}
\begin{document}
%-------------------------------------------------------------------------%
\newpage
\section{Section 3 Logic}
\subsection{Logical Operations}
\begin{itemize}
\item $\neg p$ the negation of proposition $p$.
\item $p \wedge q$ Both propositions p and q are simultaneously true (Logical State AND)
\item $p \vee q $ One of the propositions is true, or both (Logical State : OR)
\item $p \otimes q$ Only one of the propositions is true (Logical State : exclusive OR (i.e XOR)
\end{itemize}
\begin{center}
\begin{tabular}{|c|c|c|c|c|}
\hline
p & q & $p \vee q$ & $q \wedge p$ & $p \otimes q$ \\
\hline
0 & 0 & 0 & 0 & 0 \\
0 & 1 & 1 & 0 & 1\\
1 & 0 & 1 & 0 & 1 \\
1 & 1 & 1 & 1 & 0\\
\hline
\end{tabular}
\end{center}
%---------------------------------------------------------%
\section{Conditional Connectives}
Construct the truth table for the proposition $p \rightarrow q$.

\begin{center}
\begin{tabular}{|c|c|c|c|}
\hline
p & q & $p \rightarrow q$ & $q \rightarrow p$ \\
\hline
0 & 0 & 1& 1 \\
0 & 1 & 1 & 0 \\
1 & 0 & 0 & 1 \\
1 & 1 & 1 & 1 \\
\hline
\end{tabular}
\end{center}


%-------------------------------------------------------------------------%
\newpage

x

%--------------------------------------%

\section{Logic Proposition}

Let $p$, $q$ and $r$ be the following propositions concerning integers $n$ (where $n>1$):

\begin{itemize}
\item $p$ : n is a prime factor of 36 %(2)
\item $q$ : n is a prime factor of 4 %(2)
\item $r$ : n is a prime factor of 9 %(3)
\end{itemize}
\end{frame}

\begin{center}
\begin{tabular}{|c||c|c|c|}
\hline 
\phantom{spa} \textbf{n} \phantom{spa}	& \phantom{spa}	\textbf{p} \phantom{spa}	& \phantom{spa}	\textbf{q} \phantom{spa}	& \phantom{spa}	\textbf{r} \phantom{spa}	\\ \hline \hline
1	&	1	&	1	&	1	\\ \hline
2	&	1	&	0	&	1	\\ \hline
3	&	0	&	1	&	1	\\ \hline
%4	&	1	&	0	&	1	\\ \hline
%6	&	0	&	0	&	1	\\ \hline
%9	&	0	&	1	&	1	\\ \hline
%12	&	0	&	0	&	1	\\ \hline
%18	&	0	&	0	&	1	\\ \hline
%36	&	0	&	0	&	1	\\
\hline 
\end{tabular} 
\end{center}
\end{frame}
%------------------------------------------------------------------------------- %
\begin{frame}

{Logical Propostions}

For each of the following compound statements, express it using the propositions $p$,$q$ and $r$, and the appropriate logical symbols, then given the truth table for it,

\begin{itemize}
\item[1)] If n is a prime factor of 36, then n is a prime factor of 4 or n is a prime factor of 9
\item[2)] If n is a prime factor of 4 or n is a prime factor of 9, then  n is a prime factor of 36
\end{itemize}
\end{frame}
\begin{frame}
%-------------------------------------- %
{Logical Propostions}

%\subsection*{1.1 2010 Question 3}
Suppose $S = \{10,11,12,13,14,15,16,17,18,19\}$.
%% - \vspace{0.3cm}

Let p, q be the following propositions concerning the integer $n \in S$.
%% - \vspace{0.3cm}
\begin{itemize}
\item[p:] n is a multiple of two. \\(i.e. $\{10,12,14,16,18\}$)
%% - \vspace{0.3cm}
\item[q:] n is a multiple of three. \\(i.e. $\{12,15,18\}$)
\end{itemize}

For each of the following compound statements find the sets of values n for which it is true. 

\begin{itemize}
\item[(i)] $p \vee q$ : (\textit{p or q}) :  $\{10, 12, 14, 15, 16, 18\}$ 
%% - \vspace{0.5cm}
\item[(ii)] $p \wedge q$: (\textit{p and q}):  $\{12, 18\}$
%% - \vspace{0.5cm}
\item[(iii)] $ p \oplus q$: (\textit{p or q, but not both}) :  $\{10, 14, 15, 16\}$
\end{itemize}

Recall p = $\{10,12,14,16,18\}$  and q = $\{12,15,18\}$


For each of the following compound statements find the sets of values n for which it is true. %% - \vspace{0.3cm}
\begin{itemize}
\item[(iv)] $\neg p $  (i.e. \textit{not-p}) %= $\{ 11, 13, 15, 17, 19\}$
%% - \vspace{0.3cm}
\item[(v)] $\neg p \vee q$  (i.e. \textit{not-p or q}) % =  $\{11, 12, 13, 15, 17, 18, 19\}$
\end{itemize}

Recall S= \{10, 11, 12, \ldots ,18, 19\}, p = $\{10,12,14,16,18\}$  and q = $\{12,15,18\}$

%------------------------------------------------------------------ %



\begin{center}
\begin{tabular}{|c|c|c|c|c|}
\hline
p & q & $p \vee q$ & $q \wedge p$ & $p \otimes q$ \\
\hline
0 & 0 & 0 & 0 & 0 \\
0 & 1 & 1 & 0 & 1\\
1 & 0 & 1 & 0 & 1 \\
1 & 1 & 1 & 1 & 0\\
\hline
\end{tabular}
\end{center}


%------------------------------------------------------------------ %

\section{Logical Propositions}

For each of the following compound statements find the sets of values n for which it is true. %% - \vspace{0.3cm}
\begin{itemize}
\item[(iv)] $\neg p $  (i.e. \textit{not-p}) % = $\{ 11, 13, 15, 17, 19\}$
%% - \vspace{0.3cm}
\item[(v)] $\neg p \vee q$  (i.e. \textit{not-p or q}) % =  $\{11, 12, 13, 15, 17, 18, 19\}$
\end{itemize}
%================================================================%
Recall S= \{10, 11, 12, \ldots ,18, 19\}, p = $\{10,12,14,16,18\}$  and q = $\{12,15,18\}$

For each of the following compound statements find the sets of values n for which it is true. %% - \vspace{0.3cm}
\begin{itemize}
\item[(iv)] $\neg p $  (i.e. \textit{not-p}) = $\{ 11, 13, 15, 17, 19\}$
%% - \vspace{0.3cm}
\item[(v)] $\neg p \vee q$  (i.e. \textit{not-p or q}) =  $\{11, 12, 13, 15, 17, 18, 19\}$
\end{itemize}
%% - \vspace{0.7cm}

Recall S= \{10, 11, 12, \ldots ,18, 19\}, p = $\{10,12,14,16,18\}$  and q = $\{12,15,18\}$



For each of the following compound statements find the sets of values n for which it is true. 
%% - \vspace{0.4cm}
\begin{itemize}
\item[(vi)] $\neg p \wedge q$ (i.e. \textit{not-p and q}) = $\{15\} $
%% - \vspace{0.5cm}
\item[(vii)] $ \neg p \oplus q$ (i.e. \textit{ not-p or q but not both})= $\{11, 12, 13, 17, 18, 19\}$
\end{itemize}
%% - \vspace{0.7cm}

Recall S= \{10, 11, 12, \ldots ,18, 19\}, $\neg$p = $\{ 11, 13, 15, 17, 19\}$  and q = $\{12,15,18\}$

For each of the following compound statements find the sets of values n for which it is true. 
%% - \vspace{0.4cm}
\begin{itemize}
\item[(vi)] $\neg p \wedge q$ (i.e. \textit{not-p and q}) % = $\{15\} $
%% - \vspace{0.5cm}
\item[(vii)] $ \neg p \oplus q$ (i.e. \textit{ not-p or q but not both}) %= $\{11, 12, 13, 17, 18, 19\}$
\end{itemize}
%% - \vspace{0.7cm}

Recall S= \{10, 11, 12, \ldots ,18, 19\}, $\neg$p = $\{ 11, 13, 15, 17, 19\}$  and q = $\{12,15,18\}$


{Proof With Truth Tables}

Let $p$ and $q$ be propositions. Use\textbf{\textit{Truth Tables}} to prove that

\[ p \rightarrow q \equiv \neg q \rightarrow \neg p\]
\end{frame}
\begin{frame}
%-------------------------------------- %
{Proof With Truth Tables}

%% - \vspace{-1cm}
\textbf{Important}\\ 
Remember to make a comment at the end to say why the table proves that the two statements are logically equivalent. \\ For example : \emph{``Since the relevant columns are identical, then it can be said that both sides of the equation are equivalent"}.
\end{frame}
\begin{frame}
%-------------------------------------- %
{Proof With Truth Tables}

%% - \vspace{-1cm}
Left hand side of expression : $p$ \textit{implies} $q$.
\[p \rightarrow q\]
\begin{center}
\begin{tabular}{|c|c||c|}
\hline  \phantom{spa}p\phantom{spa}&  \phantom{spa}q\phantom{spa}& \phantom{sp}$p \rightarrow q$ \phantom{sp} \\ 
\hline  0&  0&  1\\ 
\hline  0&  1&  1\\ 
\hline  1&  0&  0\\ 
\hline  1&  1&  1\\ 
\hline 
\end{tabular} 
\end{center}



{Proof With Truth Tables}

%% - \vspace{-1cm}
Right hand side of expression : \textit{not-q} \textit{implies} \textit{not-p}
\[\neg q \rightarrow \neg p\]
\begin{center}
\begin{tabular}{|c|c||c|c|c|}
\hline  \phantom{sp}p\phantom{sp}&  \phantom{sp}q\phantom{sp}&\phantom{sp} $\neg q$ \phantom{sp} & \phantom{sp} $\neg p \phantom{sp}$ & $\neg q \rightarrow \neg p$ \\ 
\hline  0&  0& 1& 1& 1\\ 
\hline  0&  1& 0& 1& 1\\ 
\hline  1&  0& 1& 0& 0\\ 
\hline  1&  1& 0& 0& 1\\ 
\hline 
\end{tabular}
\end{center}


Side by Side
\[ p \rightarrow q \equiv \neg q \rightarrow \neg p\]
\bigskip
{ 
\hspace{0.5cm} \begin{tabular}{|c|c||c|}
\hline  p&  q& $p \rightarrow q$ \\ 
\hline  0&  0&  1\\ 
\hline  0&  1&  1\\ 
\hline  1&  0&  0\\ 
\hline  1&  1&  1\\ 
\hline 
\end{tabular} \hspace{0.5cm} \begin{tabular}{|c|c||c|c|c|}
\hline  p&  q& $\neg q$ & $\neg p$ & $\neg q \rightarrow \neg p$ \\ 
\hline  0&  0& 1& 1& 1\\ 
\hline  0&  1& 0& 1& 1\\ 
\hline  1&  0& 1& 0& 0\\ 
\hline  1&  1& 0& 0& 1\\ 
\hline 
\end{tabular}
}\\
%% - \vspace{0.5cm}
(only ``difference" is first and last rows)
\end{frame}

%--------------------------------------------------- %
%--------------------------------------------------- %
%\subsection*{1.3 Membership Tables for Laws}
%\emph{Page 44 (Volume 1) Q8.
%Also see Section 3.3 Laws of Logic.}\\
%--------------------------------------%


\begin{frame}
\Huge
\[\mbox{Discrete Maths :  Logic}\]
\[\mbox{Laws of Logic}\]
\bigskip

\[\mbox{www.Stats-Lab.com}\]
\[\mbox{Twitter: @StatsLabDublin}\]
\end{frame}

\begin{frame}

{Laws of Logic}

%% - \vspace{-1cm}
Construct a truth table for each of the following compound statement and hence find simpler propositions to which it is equivalent.


\begin{itemize}
\item[(i)] $p \vee F$
\item[(ii)] $p \wedge T$
\end{itemize}
\end{frame}
%--------------------------------------------------- %
%--------------------------------------------------- %
\begin{frame}
{Laws of Logic}

%% - \vspace{-1cm}
\textbf{Solutions}
\begin{center}

\begin{tabular}{|c|c||c||c|}
\hline  \phantom{sp}p\phantom{sp}&  \phantom{sp}T\phantom{sp}& $p \vee T$ & $ p \wedge T$ \\ \hline
\hline  0 & 1 &  &  \\ 
\hline  1 &  1 &  &  \\ 
\hline 
\end{tabular} 
\end{center}
\end{frame}

%--------------------------------------------------- %
%--------------------------------------------------- %
\begin{frame}
{Laws of Logic}

%% - \vspace{-1cm}
\textbf{Solutions}
\begin{center}

\begin{tabular}{|c|c||c||c|}
\hline  \phantom{sp}p\phantom{sp}&  \phantom{sp}T\phantom{sp}& $p \vee T$ & $ p \wedge T$ \\ \hline
\hline  0 & 1 & 1 & 0 \\ 
\hline  1 &  1 & 1 & 1 \\ 
\hline 
\end{tabular} 
\end{center}
\begin{itemize}
\item[(i)] $p \vee F \equiv T$
\item[(ii)] $p \wedge T \equiv p$
\end{itemize}
\end{frame}

\begin{frame}

{Laws of Logic}

%% - \vspace{-1cm}
Construct a truth table for each of the following compound statement and hence find simpler propositions to which it is equivalent.


\begin{itemize}
\item[(iii)] $p \vee F$
\item[(iv)] $p \wedge F$
\end{itemize}
\end{frame}
%--------------------------------------------------- %
%--------------------------------------------------- %
\begin{frame}
{Laws of Logic}

%% - \vspace{-1cm}
\textbf{Solutions}
\begin{center}

\begin{tabular}{|c|c||c|c|}
\hline  \phantom{sp}p\phantom{sp}&  \phantom{sp}F\phantom{sp}& $p \vee F$ & $ p \wedge F$ \\ \hline
\hline  0 & 0 &  &  \\ 
\hline  1 &  0 &  &  \\ 
\hline 
\end{tabular} 

\end{center}
\end{frame}
%--------------------------------------------------- %
%--------------------------------------------------- %
\begin{frame}
{Laws of Logic}

%% - \vspace{-1cm}
\textbf{Solutions}
\begin{center}
\begin{tabular}{|c|c||c|c|}
\hline  \phantom{sp}p\phantom{sp}&  \phantom{sp}F\phantom{sp}& $p \vee F$ & $ p \wedge F$ \\ \hline
\hline  0 & 0 & 0 & 0 \\ 
\hline  1 &  0 & 1 & 0 \\ 
\hline 
\end{tabular} 

\end{center}
\begin{itemize}
\item[(iii)] $p \vee F = p $
\item[(iv)] $p \wedge F = F $
\end{itemize}
\end{frame}
%--------------------------------------------------- %
%--------------------------------------------------- %
%\begin{frame}
%{Laws of Logic}
%
%\begin{itemize}
%\item Logical OR:  $p \vee F = p $
%\item Logical AND: $p \wedge F = F $
%\end{itemize}
%\end{frame}

\begin{frame}
\Huge
\[\mbox{Discrete Maths :  Logic}\]
\[\mbox{Contra-positives}\]
\bigskip

\[\mbox{www.Stats-Lab.com}\]
\[\mbox{Twitter: @StatsLabDublin}\]
\end{frame}


%--------------------------------------------------- %
%--------------------------------------------------- %
%\subsection*{1.4 Propositions}
%\textbf{Page 67 Question 9}
\begin{frame}
{Contra-positive}

Write the contra-positive of each of the following statements:

\begin{itemize}
\item If n= 12, then n is divisible by 3.
\item If n=5, then n is positive.
\item If the quadrilateral is square, then four sides are equal.
\end{itemize}
\end{frame}
\begin{frame}
{Contra-positives}

\textbf{Solutions}
\begin{itemize}
\item If n is not divisible by 3, then n is not equal to 12.
\item If n is not positive, then n is not equal to 5.
\item If the four sides are not equal, then the quadrilateral is not a square.
\end{itemize}
\end{frame}

%--------------------------------------------------- %
%--------------------------------------------------- %
\begin{frame}
\Huge
\[\mbox{Discrete Maths :  Logic}\]
\[\mbox{Truth Sets}\]
\bigskip

\[\mbox{www.Stats-Lab.com}\]
\[\mbox{Twitter: @StatsLabDublin}\]
\end{frame}


%--------------------------------------------------- %
%--------------------------------------------------- %
%\subsection*{1.5 Truth Sets}
\begin{frame}
{Truth Sets}


\textbf{2009} 

Let $n = \{1, 2,3,4, 5,6,7, 8, 9\}$ and let $p$ and  $q$ be the following propositions concerning the integer $n$.
\begin{itemize}
\item p: n is even, 
\item q: $n\geq 5$.
\end{itemize}
By drawing up the appropriate truth table find the truth set for each of the
propositions $p \vee \neg q$ and $ \neg q \rightarrow p$
\end{frame}

%--------------------------------------------------- %
%--------------------------------------------------- %
%\subsection*{1.5 Truth Sets}
\begin{frame}
{Truth Sets}

%% - \vspace{-0.5cm}
\begin{center}
\begin{tabular}{|c||c|c||c||c|}
\hline \phantom{sp} n \phantom{sp} & \phantom{sp} p \phantom{sp} & \phantom{sp}q \phantom{sp}& \phantom{s} $\neg q$ \phantom{s}& \phantom{s} $p \vee \neg q$ \phantom{s}\\  \hline
\hline 1 & 0 & 0 & 1 & \\ 
\hline 2 & 1 & 0 & 1 & \\ 
\hline 3 & 0 & 0 & 1 & \\ 
\hline 4 & 1 & 0 & 1 & \\ 
\hline 5 & 0 & 1 & 0 & \\ 
\hline 6 & 1 & 1 & 0 & \\ 
\hline 7 & 0 & 1 & 0 & \\ 
\hline 8 & 1 & 1 & 0 & \\ 
\hline 9 & 0 & 1 & 0 & \\ 
\hline 
\end{tabular}
\end{center} 
%\[\mbox{Truth Set} = \{1,3,5,6,7,8,9\}\]
\end{frame}
%--------------------------------------------------- %
%--------------------------------------------------- %
%\subsection*{1.5 Truth Sets}
\begin{frame}
{Truth Sets}

%% - \vspace{-0.5cm}
\begin{center}
\begin{tabular}{|c||c|c||c||c|}
\hline \phantom{sp} n \phantom{sp} & \phantom{sp} p \phantom{sp} & \phantom{sp}q \phantom{sp}& \phantom{s} $\neg q$ \phantom{s}& \phantom{s} $p \vee \neg q$ \phantom{s}\\  \hline
\hline 1 & 0 & 0 & 1 & 1\\ 
\hline 2 & 1 & 0 & 1 & 0\\ 
\hline 3 & 0 & 0 & 1 & 1\\ 
\hline 4 & 1 & 0 & 1 & 0\\ 
\hline 5 & 0 & 1 & 0 & 1\\ 
\hline 6 & 1 & 1 & 0 & 1\\ 
\hline 7 & 0 & 1 & 0 & 1\\ 
\hline 8 & 1 & 1 & 0 & 1\\ 
\hline 9 & 0 & 1 & 0 & 1\\ 
\hline 
\end{tabular}
\end{center} 
\[\mbox{Truth Set} = \{1,3,5,6,7,8,9\}\]
\end{frame}

%--------------------------------------------------- %
%--------------------------------------------------- %
%\subsection*{1.5 Truth Sets}
\begin{frame}
{Truth Sets}

\begin{center}
\begin{tabular}{|c||c|c||c||c|}
\hline \phantom{sp} n \phantom{sp} & \phantom{sp} p \phantom{sp} & \phantom{sp}q \phantom{sp}& \phantom{s} $  p \rightarrow q$ \phantom{s}& \phantom{s} $q \rightarrow p$ \phantom{s}\\  \hline
\hline 1 & 0 & 0 & 1 & 0\\ 
\hline 2 & 1 & 0 & 1 & 0\\ 
\hline 3 & 0 & 0 & 1 & 0\\ 
\hline 4 & 1 & 0 & 1 & 0\\ 
\hline 5 & 0 & 1 & 0 & 1\\ 
\hline 6 & 1 & 1 & 1 & 0\\ 
\hline 7 & 0 & 1 & 0 & 1\\ 
\hline 8 & 1 & 1 & 1 & 0\\ 
\hline 9 & 0 & 1 & 0 & 1\\ 
\hline 
\end{tabular} 
\end{center}
\[\mbox{Truth Set} = \{5,7,9\}\]
\end{frame}

%--------------------------------------------------- %
%--------------------------------------------------- %
%\subsection*{1.5 Truth Sets}

%--------------------------------------------------- %
%--------------------------------------------------- %
%--------------------------------------------------- %
\begin{frame}
{Biconditional}
\emph{See Section 3.2.1}.\\

Use truth tables to prove that $ \neg p \leftrightarrow \neg q $ is equivalent to  $ p \leftrightarrow q $
\begin{center}
\begin{tabular}{|c|c|c|}
\hline  p& q & $p \leftrightarrow q$ \\ 
\hline  0& 0 &  1\\ 
\hline  0& 1 &  0\\ 
\hline  1& 0 &  0\\ 
\hline  1& 1 &  1\\ 
\hline 
\end{tabular}
\end{center} 
\end{frame}
%--------------------------------------------------- %
\begin{frame}
{Biconditional}

%% - \vspace{-1cm}
\begin{tabular}{|c|c|c|c|c|}
\hline  \phantom{sp} p \phantom{sp} & \phantom{sp} q \phantom{sp} & $\neg p$ & $\neg q$ & $p \leftrightarrow q$ \\ 
\hline  0& 0 & 1& 1 & 1\\ 
\hline  0& 1 &  1& 0& 0\\ 
\hline  1& 0 &  0& 1& 0\\ 
\hline  1& 1 &  0 & 0& 1\\ 
\hline 
\end{tabular} 
\end{frame}
%--------------------------------------------------- %
%--------------------------------------------------- %
\begin{frame}
\Huge
\[\mbox{Discrete Maths :  Logic}\]
\[\mbox{Logic Networks}\]
\bigskip

\[\mbox{www.Stats-Lab.com}\]
\[\mbox{Twitter: @StatsLabDublin}\]
\end{frame}


%--------------------------------------------------- %
%--------------------------------------------------- %
\begin{frame}

{Logic Networks }
\emph{(2008 Q3b)}\\
Construct a logic network that accepts as input p and q, which may independently have the value 0 or 1, and
gives as final input $\neg(p \wedge \not q)$ (i.e. $\equiv p \rightarrow q$).\\
\end{frame}
%--------------------------------------------------- %
%--------------------------------------------------- %
\begin{frame}

{Logic Networks }

\textbf{Logic Gates}
\begin{itemize}
\item AND
\item OR
\item NOT
\end{itemize}
\end{frame}
%--------------------------------------------------- %
%--------------------------------------------------- %
\begin{frame}

\emph{\textbf{Examiner's Comments:}Many
diagrams were carefully and clearly drawn and well labelled, gaining full
marks. The logic table was also well done by most, but there were a few marks
lost in the final part by failing to deduce that ‘since the columns of the table are
identical the expressions are equivalent’.}
\end{frame}
%--------------------------------------------------- %
%--------------------------------------------------- %
\begin{frame}
%--------------------------------------------------- %
%--------------------------------------------------- %
{1.8 2008 Q3b Logic Networks }
Construct a logic network that accepts as input p and q, which may independently have the value 0 or 1, and
gives as final input $(p \wedge  q) \vee \neg q$ (i.e. $\equiv p \rightarrow q$).



\textbf{Important} Label each of the gates appropriately and label the diagram with a symblic expression for the output after each gate.

\end{frame}
%--------------------------------------------------- %




\end{document}

%--------------------------------------------%
\section{Chapter 3: Logic}

\textbf{2003 Question 3}\\
Let p, q be the following propositions:
\begin{itemize}
\item p : this apple is red, 
\item q : this apple is ripe.
\end{itemize}

Express the following statements in words as simply as you can:
\begin{itemize}
\item (i) $p \rightarrow q$
\item (ii) $p \wedge \neg q$.
\end{itemize}

 
Express the following statements symbolically:
\begin{itemize}
\item (iii) This apple is neither red nor ripe.
\item (iv) If this apple is not red it is not ripe.
\end{itemize}



\subsection{Logical Operations}

\begin{itemize}
\item Logical "AND" ($\wedge$)
\item Logical "OR" ($\vee$)
\item Logical "NOT" 
\end{itemize}
%--------------table
%\begin{tabular}{cccc}\hline
%p & q& P \vee q& p \wedge q  \\
%0& 0& 0& 0\\ 
%0& 1& 0& 1 \\ 
%1& 0& 0& 1 \\ 
%1& 1&1 &1  \\  \hline
%\end{tabular}
\newpage

%--------------------------------------------%
\subsection*{Logic Networks}
\begin{itemize}
\item AND Gates
\item OR Gates
\item NOT Gates
\end{itemize}

\newpage

%--------------------------------------------------- %
\subsection*{1.2 2010 Question 3}

Let p and q be propositions. Use Truth Tables to prove that

\[ p \rightarrow q \equiv \neg q \rightarrow \neg\]
\textbf{Important} Remember to make a comment at the end to say why the table proves that the two statements are logically equivalent. e.g. \emph{‘since the columns are identical both sides of the equation are equivalent’}.
{ \large
\begin{tabular}{|c|c||c|}
\hline  p&  q& $p \rightarrow q$ \\ 
\hline  0&  0&  1\\ 
\hline  0&  1&  1\\ 
\hline  1&  0&  0\\ 
\hline  1&  1&  1\\ 
\hline 
\end{tabular} \hspace{0.5cm} \begin{tabular}{|c|c||c|c|c|}
\hline  p&  q& $\neg q$ & $\neg p$ & $\neg q \rightarrow \neg p$ \\ 
\hline  0&  0& 1& 1& 1\\ 
\hline  0&  1& 0& 1& 1\\ 
\hline  1&  0& 1& 0& 0\\ 
\hline  1&  1& 0& 0& 1\\ 
\hline 
\end{tabular}
} 
(Key ``difference" is first and last rows)
%--------------------------------------------------- %
